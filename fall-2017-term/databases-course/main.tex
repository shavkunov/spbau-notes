
\documentclass[13pt,a4paper]{scrartcl}
\usepackage[utf8]{inputenc}
\usepackage[english,russian]{babel}
\usepackage{indentfirst}
\usepackage{graphicx}
\usepackage{amsmath}
\usepackage{amssymb}
\usepackage{listings}
\usepackage{array}
\usepackage{import}
\usepackage{listings}
\usepackage{color}
\usepackage{graphviz}
\usepackage{tabularx}
\usepackage[left=1cm,right=1cm,top=2cm,bottom=2cm]{geometry}

\definecolor{dkgreen}{rgb}{0,0.6,0}
\definecolor{gray}{rgb}{0.5,0.5,0.5}
\definecolor{mauve}{rgb}{0.58,0,0.82}

\lstset{frame=tb,
  language=C,
  aboveskip=3mm,
  belowskip=3mm,
  showstringspaces=false,
  columns=flexible,
  basicstyle={\small\ttfamily},
  numbers=left,
  stepnumber=1,
  numberstyle=\tiny\color{gray},
  keywordstyle=\color{blue},
  commentstyle=\color{dkgreen},
  stringstyle=\color{mauve},
  breaklines=true,
  breakatwhitespace=true,
  keepspaces=true,
  tabsize=3,
  extendedchars=\true,
  mathescape=true
}

% Некоторые множества

\def\Q{\mathbb{Q}}
\def\Z{\mathbb{Z}}
\def\N{\mathbb{N}}
\def\R{\mathbb{R}}
\def\C{\mathbb{C}}

% Бинарные операции над множествами

% xor
\def\xor{\oplus}
% объединение
\def\u{\cup}
% объединение
\def\i{\cap}


% Комбинаторика

% Биномиальный коэффициент : (из n  по k)
\def\set{\binom}
% ((из n по k))
\def\mset#1#2{\ensuremath{\left(\kern-.3em\left(\genfrac{}{}{0pt}{}{#1}{#2}\right)\kern-.3em\right)}}


% Опеации над несколькими множествами

% Сумма
\def\suml{\sum\limits}
\def\sumfor#1#2#3{\suml_{{#1}={#2}}^{#3}}
\def\sumi#1#2{\sumfor{#1}{#2}{\inf}}
\def\sumin#1#2{\suml_{{#1} \in {#2}}}

% Перемножение (знак П)
\def\mul{\prod}
\def\mull{\mul\limits}
\def\muli#1#2{\mull{#1}{#2}{\inf}}
\def\mulin#1#2{\mul\limits_{{#1} \in {#2}}}

% Объединение
\def\U{\bigcup}
\def\Ul#1#2#3{\U\limits_{{#1}={#2}}^{#3}}
\def\Ui#1#2{\Ul{#1}{#2}{\inf}}
\def\Uin#1#2{\U\limits_{{#1} \in {#2}}}

% Пересечение
\def\I{\bigcap}
\def\Il#1#2#3{\I\limits_{{#1}={#2}}^{#3}}
\def\Ii#1#2{\Il{#1}{#2}{\inf}}
\def\Iin#1#2{\I\limits_{{#1} \in {#2}}}


% Разделители

\def\ms{\medskip}
\def\bs{\bigskip}


% Греческий алфавит

\def\a{\alpha}
\def\b{\beta}
\def\g{\gamma}
\def\l{\lambda}
\def\e{\varepsilon}
\def\eps{\varepsilon}
\def\d{\delta}
\def\m{\mu}
\def\p{\phi}

\def\L{\Lambda}
\def\D{\Delta}
\def\M{\Mu}
\def\P{\Phi}


% Кванторы

\def\A{\forall}
\def\E{\exists\;}


% Что-то еще

\def\inf{\t{+}\infty}    % +inf
\def\O{\mathcal{O}}      %
\def\t{\text}
\def\bs{\textbackslash{}}

\begin{document}

\title{Базы данных}
\author{Миша}
\maketitle	

\section*{Лекция 1
}
\noindent Преподаватель: Дмитрий Барашев \\

\subsection*{Организационное
}
\noindent Начальный уровень. \\
\noindent Слайдов нет! Но может что-то изменится. \\

\noindent Практос: реляционные БД(SQL) \\
\noindent Лектос: теория БД и внутреннее устройство. \\

\subsubsection*{Оценки.
}
\noindent По практике можно заработать 0 или -1 балл(по сути это зачет и незачет) \\
\noindent На лекциях можно заработать [0…5]. Экзамен. Письменный. Будет какие-то задачи. Какие-то легкие, какие-то нет. \\
\noindent Оценки суммируются. \\
\noindent Получить 5 баллов очень сложно. Со слов преподавателя. \\

\subsubsection*{Список литературы.
}
\noindent Крис Дейт “Введение в базы данных” теоритические аспекты \\
\noindent Джефри Ульман, Г. Гарсиа-Молина, Уидом “Базы данных полный курс” железо, алгоритмы \\
\noindent Википедия — отстой. \\

\subsection*{Введение
}
\noindent Что хотим с данными делать: \\
\noindent - уметь обрабатывать \\
\noindent - поиск сортировка \\

\noindent Хотим добавить заголовок. Смысл. \\
\noindent Метаинформация:  \\
\noindent - тип данных. \\
\noindent - название, \\
\noindent - ограничения \\

\noindent Определение БД можно сформулировать следующим образом: интегрированое самодокументированное хранилище информации. \\

\noindent Но поговорим лучше о применении БД \\
\noindent Мы будем использовать PostgreSQL \\

\noindent С точки зрения данных: \\
\noindent Это набор файлов(а может и нет) \\
\noindent Они находятся в некоторой единой структуре(физическая организация) \\
\noindent С точки зрения кода: \\
\noindent prog.exe программа, управляющая данными. Обычно это называют СУБД. \\

\noindent Модели данных: \\
\noindent Объектно-Ориентированная \\
\noindent Традиционная(таблички) \\
\noindent Иерархические(JSON) и как продолжение этой модели — графовая \\

\noindent Что делает СУБД? \\
\noindent 1. Управляет размещением данных на диске \\
\noindent 2. Выполняет запросы \\
\noindent 3. Предоставить высокоуровневый язык для манипуляции с БД \\
\noindent OQL(object query language), SQL, MongoDB(монгодибишный язык) \\
\noindent Нельзя просто так взять и перенести из одной БД в другую данные на SQL. \\
\noindent Стандарты есть, но всем пофиг \\
\noindent 4. Разделение прав доступа.(управление ролями). Представление данных для разных ролей \\
\noindent 5. Инструменты для бэкапов \\
\noindent 6. Управление конкурентным доступом \\
\noindent 7. Восстановление после сбоев \\
\noindent Т.е. СУБД делает много. Какие виды СУБД? \\

\noindent Архитектура информационных систем с БД \\
\noindent 1. Встроенные БД.Например SQLite \\
\noindent 2. Клиент-серверная архитектура. Но есть проблема: очень сложно управлять. то есть допустим заведуем БД в городе и чтобы \\
\noindent реализовать какую-то фичу, придется обновится клиента на всех клиентских компьютерах. Если в БД используется где-нибудь в \\
\noindent госструктурах, то можно стрелятся. Даже если есть автообновление, деятельность ВСЕЙ БД приходится на какое-то время \\
\noindent останавливать.  \\
\noindent Поэтому люди задумались, чтобы код SQL жил где-нибудь поближе к ядру СУБД.  \\
\noindent 3. Многоуровневая архитектура. то есть СУБД <—> сервер приложений и уже к этому серверу направляется пользовательский ввод. \\
\noindent Поэтому в этом случае мы просто обновим сервер приложений. Эта система доминирует на данный момент в Web-приложениях. \\



\subsection*{Практика
}
\noindent Цель: получение некоторого законченного проекта. \\

\noindent Виды реляционных СУБД: \\
\noindent 1. Oracle \\
\noindent 2. IBM DB2 \\
\noindent 3. MS SQL Server \\
\noindent 4. MySQL / MariaDB \\
\noindent 5. PostgreSQL  \\

\noindent Почта: dmitry@barashev.net \\

\noindent Тут нам показывают как работать с бд \\
\noindent sqlitebrowser — простой GUI для sqlite \\


\noindent Нужно разбиться на команды по 3-4 человека и делать задания самостоятельно. Задания по практике будут командными. НО за это не \\
\noindent будут даваться баллы. \\
\noindent Зачёт/незачёт будет ставится по контрольным, которые будут уже индивидуальными. \\
\noindent Поэтому общаться с друг другом стоит. \\

\noindent К следующему занятию нужно научиться работать с PostgreSQL из терминала. \\

\section*{Водяная лекция 2
}

\subsection*{Реляционная модель
}

\noindent В ООП есть свои кирпичики — классы. \\
\noindent В реляционной модели это таблицы или отношения. \\

\noindent Основной элемент: отношение(relation) \\

\noindent Домен := некоторое множество значений(бесконечное или конечное) \\
\indent:= тип данных + некоторые ограничения на этот тип
\\
\noindent В домене определены некоторые операции. Сравнение, + * - / \\

\noindent Примеры доменов: BOOL, INT, TEXT, INT > 0 и так далее. \\


\noindent Рассмотрим следующую функцию:  \\
\noindent $f: D_1 \times
 … \times
 D_n -> true, false$ \\
\noindent $D_i$ — домен \\
\noindent $\{D_i\}_{i = 1}^n$ \\

\noindent $f$ — отношение на доменах $D_1, … D_n$. \\

\noindent Есть и другое определение: \\
\noindent Отношение: схема + тело. \\
\noindent Схема отношения — множество атрибутов. то есть $S = {(a_i, D_i)}$, $a_i$ — имя, $D_i$ — домен. Атрибут это пара. \\
\noindent $a_i == a_j$ => атрибуты равны. \\
\noindent Тело отношения — множество картежей. Кортеж $t = {(a_i, v_i)}$ $a_1, … a_n$ — последовательность имен атрибутов в $S$. \\
\noindent $v_i \in D_i$ \\

\noindent таблица№1 \\
\noindent Это графическое представление. \\

\noindent Можно также изобразить тоже самое в виде JSON: \\

\noindent \{ \\
\indent“a1”: 1,
\\\indent“a2”: true,
\\\indent“a3”: “abc”,
\\\indent“a4”: 3.14
\\\noindent \} \\

\noindent Если предикат f возращает истину, то строчка есть в таблице, иначе ее нет. \\
\noindent Одинаковые строчки мы запрещаем. \\

\noindent Равенство картежей \\
\noindent $t = t’$ если $V_{a_i}(t) = V_{a_i}(t’) \forall a_i \in S$ \\

\noindent Чего нет: \\
\noindent 1. Нет одинаковых кортежей(на практике всем пофиг) \\
\noindent 2. Нет упорядоченности кортежей \\
\noindent 3. Нет упорядоченности атрибутов \\
\noindent 4. В кортежах значений: атомарны \\

\noindent Аспекты модели данных \\

\noindent 1. Структура \\
\noindent 2. Операции \\
\noindent 3. Ограничения \\

\noindent Простейшие ограничения. \\
\noindent 1. Домен — ограничение множества значений \\
\noindent 2. Ограничения на атрибут внутри одной таблицы \\
\noindent 3. NULL или NOT NULL. \\


\noindent Потенциальный ключ. \\
\noindent Для некоторого отношения $R$ подмножство $K \subseteq
 S$ называется потенциальным ключом, если известно, что в любом экземпляре(конкретный набор строчек, то есть схема с конкретным телом) этого отношения \\
\noindent 1. Если в любом экземпляре $R$ для $t$ и $t’$ из $V_k(t) = V_k(t’) => t = t’$ \\
\noindent разных кортежей с одинаковым ключом не бывает \\
\noindent 2. (Неизбыточность) не существует $K’ 
 K$ для $K’$ выполняется 1. \\
\noindent Если выполняется только 1. то это суперключ. \\

\noindent Рассмотрим таблицу Студенты \\
\noindent направление | №паспорта | ФИО | №Студ | билета | №Группы \\

\noindent №паспорта и № студ. билета могут рассматриваться как потенциальный ключ. \\
\noindent По этим штукам можно однозначно определить студента. \\
\noindent Суперключ: например №паспорта + ФИО \\

\noindent Неизбыточность нужна, чтобы моделировать правила предметной области точно. не было студентов с совпадающими номерами паспортов \\

\noindent Потенциальный ключ (candidate key) \\
\noindent Первичный ключ(primary key) \\

\subsection*{Практика
}

\noindent stepik.org/lesson/54670 \\

\noindent И тут какие-то обычные SQL запросы. Я сижу на первом ряду и хоть немного что-то вижу, как люди на это смотрят вообще. \\

\input{build/3.tex}
\input{build/4.tex}
\input{build/5.tex}
\input{build/6.tex}
\input{build/7.tex}
\input{build/8.tex}
\input{build/9.tex}
\input{build/10.tex}
\input{build/11.tex}
\input{build/12.tex}
\input{build/13.tex}
\input{build/14.tex}
\input{build/15.tex}

\end{document}